\chapter{Estudio del Problema}

\section{Definiciones, Siglas y Abreviaciones}
En esta sección se deben listar todas las definiciones, siglas y abreviaciones relevantes para la actividad de titulación. Por ejemplo, se pueden definir en una lista como la siguiente:

\begin{itemize}
	\item ABC: Sigla que significa tal cosa.
	\item DEF: Otra sigla que significa otra cosa.
\end{itemize}

\section{Contexto del Problema}
...

\begin{figure}[H]
	\begin{center}
		\includegraphics[height=6cm]{img/logo.png}
	\end{center}
	\caption[Figura de ejemplo]{Figura de ejemplo}
	\label{fig:sample}
\end{figure}

\subsection{Diagrama de Situación Actual}
BPMN, u otro tipo de diagrama para ilustrar la situación actual.

\section{Oportunidades de Mejora o Problemáticas}
...

\section{Propuesta de Solución}
...

\section{Soluciones Similares Disponibles}
A continuación, se describen las soluciones disponibles que pueden ser catalogadas como similares al proyecto que se presenta.

\subsection{Aplicación Similar}
...

\section{Justificación del Proyecto}
...
