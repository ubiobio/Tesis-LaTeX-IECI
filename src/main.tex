\documentclass[12pt]{report}
\usepackage{amsfonts}
\usepackage{amsmath}
\usepackage{caption}
\usepackage{hyperref}
\usepackage[spanish]{cleveref}
\usepackage[utf8]{inputenc}
\usepackage[spanish, es-nodecimaldot, es-tabla]{babel}
\usepackage{geometry}
\usepackage{graphicx}
\usepackage{hyphenat}
\usepackage{listings}
\usepackage{minted}
\usepackage{multirow}
\usepackage{relsize}
\usepackage{xcolor}

\graphicspath{{img/}}

\newgeometry{
	right=3cm,
	left=3cm,
	top=2.5cm,
	bottom=2.5cm
}

\hypersetup{
	colorlinks,
	citecolor=black,
	filecolor=black,
	linkcolor=black,
	urlcolor=black
}

\newenvironment{longlisting}{\captionsetup{type=listing}}{}

% Profundidad de secciones
\setcounter{tocdepth}{4}
\setcounter{secnumdepth}{4}

% Referencias automáticas (Español)
\def\sectionautorefname{sección}
\def\subsectionautorefname{subsección}
\def\subsubsectionautorefname{subsección}
\def\figureautorefname{figura}
\def\tableautorefname{tabla}
\crefname{table}{tabla}{tablas}
\Crefname{table}{Tabla}{Tablas}

% Índice (Español)
\addto\captionsspanish{
	\renewcommand{\contentsname}%
	{Índice}%
}

\begin{titlepage}

\title{
	\begin{figure}
		\centering
			\includegraphics[width=2cm, height=3cm]{img/logo.png}\\
			{Universidad del Bío-Bío\\
			Facultad de Ciencias Empresariales\\
			Depto. de Sistemas de Información}
	\end{figure}
	{Título de la Tesis}\\
	{\large Proyecto de título para optar al título de Ingeniero de Ejecución en Computación e Informática}
}
\author{\textbf{Alumno} \\ Nombre Apellido \\\\
        \textbf{Profesor Guía} \\ Nombre Apellido \\
}
\date{Lunes 18 de diciembre, 2023}

\end{titlepage}

\begin{document}

% Título del documento
\maketitle

% Tabla de Contenidos (Índice)
\tableofcontents
\newpage

% Índice de figuras
\listoffigures
\newpage

% Índice de tablas
\listoftables
\newpage

\chapter*{Dedicatoria}
\addcontentsline{toc}{section}{Dedicatoria}
En la dedicatoria es normal referirse a la familia y a personas que hayan tenido influencia personal en torno al desarrollo profesional del tesista, o que hayan brindado apoyo moral, inspiración u otros aspectos positivos en torno a la realización del proyecto de título.


\chapter*{Agradecimientos}
\addcontentsline{toc}{section}{Agradecimientos}
En los agradecimientos, es de buena educación agradecer a los profesores que guiaron el proyecto de tesis, a la universidad del alumno tesista y a otros profesores que hayan tenido influencia en el trabajo del alumno. Además, no olvidar también incluir agradecimientos a cualquiera que haya aportado a la realización del proyecto de título.


\chapter*{Resumen}
\addcontentsline{toc}{section}{Resumen}
En el resumen se entrega una breve descripción da cada capítulo de manera individual. Por ejemplo: el capítulo 1 habla del estudio del problema, entregando detalles de su origen, características, etc. El capítulo 2 detalla los objetivos generales y específicos. Así con el resto de capítulos.


\chapter*{Introducción}
\addcontentsline{toc}{section}{Introducción}
En la introducción se describe, a grandes rasgos, el contexto general del proyecto de tesis, y una vista general de todo lo que contiene el informe de tesis.


% Estudio del Problema
\chapter{Estudio del Problema}

\section{Definiciones, Siglas y Abreviaciones}
En esta sección se deben listar todas las definiciones, siglas y abreviaciones relevantes para la actividad de titulación. Por ejemplo, se pueden definir en una lista como la siguiente:

\begin{itemize}
	\item ABC: Sigla que significa tal cosa.
	\item DEF: Otra sigla que significa otra cosa.
\end{itemize}

\section{Contexto del Problema}
...

\begin{figure}[H]
	\begin{center}
		\includegraphics[height=6cm]{img/logo.png}
	\end{center}
	\caption[Figura de ejemplo]{Figura de ejemplo}
	\label{fig:sample}
\end{figure}

\subsection{Diagrama de Situación Actual}
BPMN, u otro tipo de diagrama para ilustrar la situación actual.

\section{Oportunidades de Mejora o Problemáticas}
...

\section{Propuesta de Solución}
...

\section{Soluciones Similares Disponibles}
A continuación, se describen las soluciones disponibles que pueden ser catalogadas como similares al proyecto que se presenta.

\subsection{Aplicación Similar}
...

\section{Justificación del Proyecto}
...


% Proyecto
\chapter{Proyecto}

\section{Objetivo General del Proyecto}
Objetivo general del proyecto

\section{Objetivos Específicos del Proyecto}
\begin{enumerate}
	\item Objetivo específico
	\item Objetivo específico
	\item Objetivo específico
\end{enumerate}

\section{Metodología de Desarrollo}
Para poder definir la metodología de desarrollo a utilizar, primero se debe tener en cuenta la \autoref{table:risks}, que representa los riesgos asociados.

\begin{center}
	\begin{tabular}{|lll|l|l|ll|}
		\hline
		\multicolumn{7}{|c|}{\multirow{7}{*}\textbf{\textbf{Tabla de Riesgos}}} \\
		\hline
		\multicolumn{3}{|l|}{\multirow{2}{*}{\textbf{Experiencia en el Problema}}} &
		Alta &
		X &
		\multicolumn{2}{l|}{\multirow{2}{*}{\begin{tabular}[c]{@{}l@{}}Se tienen años de experiencia con la \\empresa.\end{tabular}}} \\ \cline{4-5}
		\multicolumn{3}{|l|}{} &
		Baja &
		&
		\multicolumn{2}{l|}{} \\ \hline
		
		\multicolumn{3}{|l|}{\multirow{2}{*}{\textbf{Tamaño del Problema}}} &
		Grande &
		X &
		\multicolumn{2}{l|}{\multirow{2}{*}{\begin{tabular}[c]{@{}l@{}}La cantidad de funcionalidades a im-\\plementar es muy alta.\end{tabular}}} \\ \cline{4-5}
		\multicolumn{3}{|l|}{} &
		Pequeño &
		&
		\multicolumn{2}{l|}{} \\ \hline
	
		\multicolumn{3}{|l|}{\multirow{2}{*}{\textbf{Complejidad del Problema}}} &
		Complejo &
		X &
		\multicolumn{2}{l|}{\multirow{2}{*}{\begin{tabular}[c]{@{}l@{}}El sistema es difícil \\de comprender y manejar completamente.\end{tabular}}} \\ \cline{4-5}
		\multicolumn{3}{|l|}{} &
		Simple &
		&
		\multicolumn{2}{l|}{} \\ \hline
		
		\multicolumn{3}{|l|}{\multirow{2}{*}{\textbf{Tamaño del Software}}} &
		Grande &
		X &
		\multicolumn{2}{l|}{\multirow{2}{*}{\begin{tabular}[c]{@{}l@{}}El software a construir require muchas\\funcionalidades.\end{tabular}}} \\ \cline{4-5}
		\multicolumn{3}{|l|}{} &
		Pequeño &
		&
		\multicolumn{2}{l|}{} \\ \hline
		
		\multicolumn{3}{|l|}{\multirow{2}{*}{\textbf{Complejidad Software}}} &
		Complejo &
		X &
		\multicolumn{2}{l|}{\multirow{2}{*}{\begin{tabular}[c]{@{}l@{}}El software debe implementar cálculos\\complejos (ratios, promedios, etc.).\end{tabular}}} \\ \cline{4-5}
		\multicolumn{3}{|l|}{} &
		Simple &
		&
		\multicolumn{2}{l|}{} \\ \hline
		
		\multicolumn{3}{|l|}{\multirow{2}{*}{\textbf{Experiencia Software}}} &
		Alta&
		X &
		\multicolumn{2}{l|}{\multirow{2}{*}{\begin{tabular}[c]{@{}l@{}}Se tiene una alta experiencia desarrollando\\software para la empresa.\end{tabular}}} \\ \cline{4-5}
		\multicolumn{3}{|l|}{} &
		Baja &
		&
		\multicolumn{2}{l|}{} \\ \hline
		
		\multicolumn{3}{|l|}{\multirow{2}{*}{\textbf{Modularidad Funcional}}} &
		Existe &
		X &
		\multicolumn{2}{l|}{\multirow{2}{*}{\begin{tabular}[c]{@{}l@{}}Las funcionalidades  pueden implementarse\\por separado y luego integrarse.\end{tabular}}} \\ \cline{4-5}
		\multicolumn{3}{|l|}{} &
		No existe&
		&
		\multicolumn{2}{l|}{} \\ \hline
	\end{tabular}%
  \\
  \captionof{table}{Tabla de Riesgos}\label{table:risks}
\end{center}

Aquí es recomendable escribir la interpretación de la tabla anterior y concluir por qué se eligió utilizar una determinada metodología.

\section{Técnicas y Notaciones}
\begin{itemize}
	\item Diagrama de Casos de Usos.
	\item BPMN para modelar el proceso de negocio actual.
	\item Carta Gantt para la planificación inicial del proyecto.
	\item Patrón de diseño MVC (Modelo, Vista, Controlador).
\end{itemize}

\section{Estándares de Documentación}
\begin{itemize}
	\item Adaptación Basada en IEEE Software Test Documentation Std 829-1998.
	\item Adaptación Basada en IEEE Software Requirements Specifications Std 830-1998.
\end{itemize}

\section{Software, Frameworks y  Lenguajes Utilizados}
\label{project:software}
A continuación se lista el software, frameworks y lenguajes de programación, marcado y estilos utilizados para la realización de este proyecto.

Para efectos del siguiente listado, los nombres de las herramientas, frameworks y lenguajes se han redactado en negrita, seguidos de paréntesis en itálica que contienen el número de la versión asociada a cada ítem.

\begin{itemize}
	\item[] \textbf{Lenguajes}
	\begin{itemize}
		\item \textbf{Ruby} \textit{(3.2.2)}: Lenguaje de programación de alto nivel.
		\item \textbf{HAML} \textit{(6.2.3)}: Lenguaje de marcado para la abstracción de HTML.
		\item \textbf{Sass} \textit{(6.0)}: Lenguaje de extensión para CSS.
	\end{itemize}
\end{itemize}

\begin{itemize}
	\item[] \textbf{Software}
	\begin{itemize}
		\item \textbf{MongoDB} \textit{(7.0.3)}: Base de datos orientada a documentos JSON.
		\item \textbf{Redis} \textit{(7.0.12)}: Almacenamiento en memoria, utilizado para el caché de datos.
		\item \textbf{RubyMine} \textit{(2023.2.2)}: Entorno de desarrollo integrado especializado para el trabajo con aplicaciones en Ruby, específicamente para Ruby on Rails.
		\item \textbf{Rake} \textit{(13.1)}: Librería de Ruby para la definición de tareas interdependientes.
		\item \textbf{MongoDB Compass} \textit{(1.39.0)}: Visor para bases de datos de MongoDB.
		\item \textbf{RedisInsight} \textit{(2.30.0)}: Visor para el almacenamiento del caché en Redis.
		\item \textbf{Docker Desktop} \textit{(4.21.0)}: Visor y gestor de contenedores de Docker, en formato de aplicación de escritorio multiplataforma.
		\item \textbf{NodeJS} \textit{(16.13.0)}: Entorno de servidor multiplataforma utilizado para la conversión de archivos en runtime.
		\item \textbf{Yarn} \textit{(1.22.21)}: Gestor de paquetes para JavaScript.
		\item \textbf{Docker} \textit{(24.0.2)}: Tecnología que permite crear y utilizar contenedores. Para efectos de este proyecto, es utilizado con el fin de probar el software desarrollado en distribuciones de Linux determinadas.
		\item \textbf{Termius} \textit{(8.7.2)}: Cliente SSH.
		\item \textbf{Git/Git Bash} \textit{(2.34.1)}: Sistema de control de versiones.
		\item \textbf{Ubuntu LTS} \textit{(18.04.6)}: Subsistema de Linux para Windows.
	\end{itemize}
\end{itemize}

\begin{itemize}
	\item[] \textbf{Frameworks}
	\begin{itemize}
		\item \textbf{Ruby on Rails} \textit{(7.1)}: Framework para desarrollo de aplicaciones web fullstack.
		\item \textbf{Jekyll} \textit{(4.0.0)}: Framework para desarrollo de aplicaciones web estáticas escrito en Ruby.
		\item \textbf{Bootstrap} \textit{(4.4.1)}: Framework para la creación de estilos, manejo de elementos visuales y la responsividad en aplicaciones web.
	\end{itemize}
\end{itemize}


% Factibilidad
\chapter{Factibilidad}

\section{Factibilidad Técnica}
\subsection{Conocimientos de los Usuarios}
¿Qué grado de conocimiento tienen los usuarios con respecto al contexto del problema? ¿Cómo se capacitará a los usuarios? ¿Tendrán disposición al cambio?

\subsection{Disponibilidad Profesional}
¿Se necesita del trabajo de un profesional del desarrollo de software u otra área específica? ¿Qué equipamiento se requiere y con qué equipamiento se cuenta? ¿Qué software se requiere y con qué tipo de acceso a él se cuenta?

% Equipamiento
\begin{center}
	\begin{tabular}{ | l | p{10cm} |}
		\hline
		\multicolumn{2}{|c|}{\textbf{Equipamiento}} \\
		\hline
		\multicolumn{1}{|c|}{\textbf{Nombre}} & \multicolumn{1}{|c|}{\textbf{Acceso}} \\
		\hline
		{\textbf{Windows PC}} & Equipo personal \\ \hline
		
		{\textbf{MacBook Pro M1}} & Equipo personal \\ \hline
	\end{tabular}
  \captionof{table}{Tabla de equipos físicos}\label{table:physical-equipment}
\end{center}

% Software
\begin{center}
	\begin{tabular}{ | l | p{10cm} |}
		\hline
		\multicolumn{2}{|c|}{\textbf{Software}} \\
		\hline
		\multicolumn{1}{|c|}{\textbf{Nombre}} & \multicolumn{1}{|c|}{\textbf{Acceso}} \\
		\hline
		
		{\textbf{Notepad++}} & Software libre \\ \hline
		
		{\textbf{MongoDB Compass}} & Software libre \\ \hline
		
		{\textbf{RedisInsight}} & Software libre \\ \hline
		
		{\textbf{Git/Git Bash}} & Software libre \\ \hline
		
		{\textbf{Ubuntu LTS}} & Software libre \\ \hline
		
		{\textbf{RubyMine}} & Licencia de estudiante \\ \hline
		
		{\textbf{Termius}} & Licencia de estudiante \\ \hline
		
		{\textbf{Microsoft  Excel}} & Licencia de estudiante \\ \hline
	\end{tabular}
  \\
  \captionof{table}{Tabla de software}\label{table:software}
\end{center}

\subsection{Despliegue y Servidor}
Características del servidor de despliegue, cómo y por qué se eligió, etc.

% Servidor
\begin{center}
	\begin{tabular}{ | l | p{10cm} |}
		\hline
		\multicolumn{2}{|c|}{\textbf{VPS}} \\
		\hline
		\multicolumn{1}{|c|}{\textbf{Característica}} & \multicolumn{1}{|c|}{\textbf{Detalle}} \\
		\hline
		
		{\textbf{Proveedor}} & DigitalOcean \\ \hline
		
		{\textbf{Región}} & Nueva York \\ \hline
		
		{\textbf{Sistema Operativo}} & Ubuntu 22.04 (LTS) x64 \\ \hline
		
		{\textbf{Tipo de CPU}} & Intel Regular \\ \hline
		
		{\textbf{Número de vCPUs}} & 1 CPU\\ \hline
		
		{\textbf{Memoria}} & 2 GB \\ \hline
		
		{\textbf{Almacenamiento (SSD)}} & 50 GB \\ \hline
		
		{\textbf{Transferencia}} & 2 TB \\ \hline
	\end{tabular}
  
  \captionof{table}{Tabla de VPS}\label{table:vps}
\end{center}

Concluir si el proyecto es factible técnicamente explicando por qué lo es.

\section{Factibilidad Operativa}
¿Los usuarios tienen disposición al cambio? ¿Los usuarios verán una mejora en su experiencia? ¿Por qué?

\section{Factibilidad Económica}
Todo lo que tiene que ver con los flujos de caja, cálculo del VAN proyectado a cierta cantidad de años, etc.

\subsection{Tablas de Costos}
\label{feasibility:costs}
Tablas de costos relacionados con el proyecto. A continuación hay algunas tablas de ejemplo:

% Costos de Producción
\begin{center}
	\begin{tabular}{ | l | p{5cm} | p{5cm}|}
		\hline
		\multicolumn{3}{|c|}{\textbf{Software}} \\
		\hline
		\multicolumn{1}{|c|}{\textbf{Nombre}} & \multicolumn{1}{|c|}{\textbf{Acceso}} & \multicolumn{1}{|c|}{\textbf{Precio (anual)}} \\
		\hline
		{\textbf{Notepad++}} & Software libre & \multicolumn{1}{|r|}{\$0} \\ \hline

		{\textbf{MongoDB Compass}} & Software libre & \multicolumn{1}{|r|}{\$0} \\ \hline
		
		{\textbf{RedisInsight}} & Software libre & \multicolumn{1}{|r|}{\$0} \\ \hline
		
		{\textbf{Git/Git Bash}} & Software libre & \multicolumn{1}{|r|}{\$0} \\ \hline
		
		{\textbf{Ubuntu LTS}} & Software libre & \multicolumn{1}{|r|}{\$0} \\ \hline
		
		{\textbf{RubyMine}} & Licencia de estudiante & \multicolumn{1}{|r|}{\$0} \\ \hline
		
		{\textbf{Termius}} & Licencia de estudiante & \multicolumn{1}{|r|}{\$0} \\ \hline
		
		{\textbf{Microsoft Excel}} & Licencia de estudiante & \multicolumn{1}{|r|}{\$0} \\ \hline
	\end{tabular}

  \captionof{table}{Tabla de costos de software}\label{table:costs:software}
\end{center}

% Costos de Producción
\begin{center}
	\begin{tabular}{ | l | p{5cm} | p{5cm}|}
		\hline
		\multicolumn{3}{|c|}{\textbf{Costos de Producción}} \\
		\hline
		\multicolumn{1}{|c|}{\textbf{Nombre}} & \multicolumn{1}{|c|}{\textbf{Proveedor}} & \multicolumn{1}{|c|}{\textbf{Precio (anual)}} \\
		\hline
		{\textbf{VPS}} & DigitalOcean & \multicolumn{1}{|r|}{\$125.000} \\ \hline
		
		{\textbf{Mailer}} & Postmark & \multicolumn{1}{|r|}{\$156.000} \\ \hline
		
		{\textbf{Dominio}} & Namecheap & \multicolumn{1}{|r|}{\$22.600} \\ \hline
    
		{\textbf{Git/Git Bash}} & Software libre & \multicolumn{1}{|r|}{\$0} \\ \hline
		
		{\textbf{Ubuntu LTS}} & Software libre & \multicolumn{1}{|r|}{\$0} \\ \hline
    
    	{\textbf{Crowdin}} & Licencia Open Source & \multicolumn{1}{|r|}{\$0} \\ \hline
    
    	{\textbf{Sentry}} & Licencia de estudiante & \multicolumn{1}{|r|}{\$0} \\ \hline

		{\textbf{RubyMine}} & Licencia de estudiante & \multicolumn{1}{|r|}{\$0} \\ \hline
	
		{\textbf{Termius}} & Licencia de estudiante & \multicolumn{1}{|r|}{\$0} \\ \hline
		
      	{\textbf{Microsoft Excel}} & Licencia de estudiante & \multicolumn{1}{|r|}{\$0} \\ \hline
	\end{tabular}

  \captionof{table}{Tabla de costos de producción}\label{table:costs:production}
\end{center}

\subsection{Flujo de Caja}
...

\subsubsection{Contexto e Indicadores Económicos}
 ...
 
 La inflación promedio anual reportada por el Banco Central de Chile \cite{bancochile}, y la prima de riesgo asociada a proyectos tecnológicos reportada por el Standish Group International en el CHAOS Manifesto del 2011 \cite{big2011chaos}.
 
 ...

% Tasa Anual
\begin{center}
	\begin{tabular}{ | p{7cm} | p{5cm}|}
		\hline
		{\textbf{Inflación Promedio Anual}} & 4.3\%  \\ \hline
		{\textbf{Tasa Prima de Riesgo}} & 21\% \\ \hline
	\end{tabular}

  \captionof{table}{Tabla de indicadores económicos}\label{table:indicators}
\end{center}

\[
\mathlarger{
  TMAR = 0.04+0.21+(0.04\cdot0.21) = 0.2584 \approx 26\%
}
\]

\subsubsection{Puesta en Marcha}
En esta sección se puede buscar una referencia del sueldo mensual promedio de un desarrollador de software, analista programador o similar. Luego, a partir de esa referencia, calcular el valor por hora y ajustarlo según las horas de trabajo efectivo.

Ejemplo: (\$900.000 mensual; \$5.538 por hora), ajustado a las horas de trabajo efectivas empleadas en el proyecto, las cuales fueron 4 horas de trabajo efectivo durante 6 días de la semana por mes (4 semanas) de desarrollo.

\[
\mathlarger{
	Desarrollo = \$5.538 \cdot \left(4 \textit{ horas}\cdot6 \textit{ días}\cdot4 \textit{ semanas}\right) = \$531.648
}
\]

% Costos Mensuales
\begin{center}
	\begin{tabular}{ | p{5cm} | p{5cm} | }
		\hline
    \multicolumn{2}{|c|}{\textbf{Costos}} \\
		\hline
		{Desarrollo} & \multicolumn{1}{|r|}{\$531.648} \\ \hline
		
		{Internet} & \multicolumn{1}{|r|}{\$10.000} \\ \hline
		
		{Electricidad} & \multicolumn{1}{|r|}{\$20.600} \\ \hline
    
    {\textbf{Total}} & \multicolumn{1}{|r|}{\textbf{\$561.648}} \\ \hline
	\end{tabular}

  \captionof{table}{Tabla de costos}\label{table:costs}
\end{center}

Una vez calculado el total de gastos para la puesta en marcha, es posible extrapolar a 4 meses y calcular la inversión inicial del proyecto a través de la siguiente fórmula.

\[
\mathlarger{
	{I_0} = \$561.648 \cdot 4\textit{ meses}= \$2.246.592
}
\]

\subsubsection{Cálculo del Valor Actual Neto}
Por ejemplo, al tratarse ahora del mantenimiento del software, las horas disminuyen, por lo que el costo también decrece:

\[
\mathlarger{
	Mantenimiento = \$5.538 \cdot \left(1 \textit{ hora}\cdot5 \textit{ días}\cdot4 \textit{ semanas}\right) = \$110.760
}
\]

\[
\frac{Desarrollo}{Mantenimiento} = \frac{\$531.648}{\$110.760} = 4.8
\]

\[
Internet = \frac{\$10.000}{4.8} = \$2.083
\]

\[
Electricidad = \frac{\$20.600}{4.8} = \$5.208
\]

% VAN Mantenimiento
\begin{center}
	\begin{tabular}{ | l | l | l | l | l | l | l |}
		\hline
		& \multicolumn{1}{|c|}{\textbf{0}} & \multicolumn{1}{|c|}{\textbf{1}} & \multicolumn{1}{|c|}{\textbf{2}} & \multicolumn{1}{|c|}{\textbf{3}} & \multicolumn{1}{|c|}{\textbf{4}} & \multicolumn{1}{|c|}{\textbf{5}} \\
		\hline
		{\textbf{Mantenimiento}} &  & \multicolumn{1}{|r|}{\$1.329.120} & \multicolumn{1}{|r|}{\$1.329.120} & \multicolumn{1}{|r|}{\$1.329.120} & \multicolumn{1}{|r|}{\$1.329.120} & \multicolumn{1}{|r|}{\$1.329.120} \\ \hline
		
		{\textbf{Internet}} &  & \multicolumn{1}{|r|}{\$24.996} & \multicolumn{1}{|r|}{\$24.996} & \multicolumn{1}{|r|}{\$24.996} & \multicolumn{1}{|r|}{\$24.996} & \multicolumn{1}{|r|}{\$24.996} \\ \hline
		
		{\textbf{Electricidad}} &  & \multicolumn{1}{|r|}{\$62.496} & \multicolumn{1}{|r|}{\$62.496} & \multicolumn{1}{|r|}{\$62.496} & \multicolumn{1}{|r|}{\$62.496} & \multicolumn{1}{|r|}{\$62.496} \\ \hline
		
		{\textbf{Hosting}} &  & \multicolumn{1}{|r|}{-\$126.312} & \multicolumn{1}{|r|}{-\$126.312} & \multicolumn{1}{|r|}{-\$126.312} & \multicolumn{1}{|r|}{-\$126.312} & \multicolumn{1}{|r|}{-\$126.312} \\ \hline
		
		{\textbf{Flujo}} &  & \multicolumn{1}{|r|}{\$1.290.300} & \multicolumn{1}{|r|}{\$1.290.300} & \multicolumn{1}{|r|}{\$1.290.300} & \multicolumn{1}{|r|}{\$1.290.300} & \multicolumn{1}{|r|}{\$1.290.300} \\ \hline
		{\textbf{Inv. Inicial}} & \multicolumn{1}{|r|}{\$2.246.592} & & & & & \\ \hline
		\textbf{Flujo Total} & \multicolumn{1}{|r|}{\$6.451.500} & & & & & \\ \hline
	\end{tabular}

  \captionof{table}{Tabla de cálculo de VAN}\label{table:van}
\end{center}

\[
\mathlarger{
	{VAN} = I_0 + \sum\limits_{t=i}^{n}\frac{C_t}{\left(1+r^t\right)} = \$5.646.624
}
\]

\section{Conclusión de Factibilidad}
Concluir sobre cada tipo de factibilidad y entregar una conclusión general del capítulo.


% Requerimientos del Software
\chapter{Requerimientos del Software}

\section{Límites}

\begin{itemize}
	\item El software no permitirá...
\end{itemize}

\section{Caracterización de los Usuarios}
Describir a quienes apunta este proyecto. Se caracterizan por:

\begin{itemize}
	\item Pertenencia a un grupo etario entre...
	\item ¿Competencias técnicas en uso de software similares?
	\item Familiaridad con...
\end{itemize}

\section{Objetivo General del Software}
Describir el objetivo general del software.

\subsection{Objetivos Específicos del Software}
\begin{itemize}
	\item Objetivo específico
	\item Objetivo específico
	\item Objetivo específico
\end{itemize}

\section{Requerimientos Funcionales del Software}
A continuación, en las tablas x a y, se definen los requerimientos funcionales del software.
\begin{center}
	\begin{tabular}{ | l | p{15cm} |}
		\hline
		\multicolumn{2}{|c|}{\textbf{Módulo de Ejemplo}} \\
		\hline
		\multicolumn{1}{|c|}{\textbf{Id}} & \multicolumn{1}{|c|}{\textbf{Descripción}} \\
		\hline
		{\textbf{RF\_01}} & La plataforma contará con un módulo de creación... \\ \hline

		{\textbf{RF\_02}} & La plataforma contará con un módulo de visualización... \\ \hline

		{\textbf{RF\_03}} & La plataforma contará con un módulo de edición de un... \\ \hline
		
		{\textbf{RF\_04}} & La plataforma contará con un módulo de eliminación de un... \\ \hline
		
		{\textbf{RF\_05}} & La plataforma contará con un módulo de visualización de un... \\ \hline
	\end{tabular}
  
  \captionof{table}{Tabla de requerimientos funcionales del módulo de ejemplo}\label{table:rf:ejemplo}
\end{center}

\section{Requerimientos No Funcionales del Software}
La presente sección hablará de los requerimientos no funcionales del software desarrollado. Todos los requerimientos no funcionales se relacionarán con uno o más atributos. Si un atributo aplica a un requerimiento no funcional, eso quiere decir que el requerimiento contribuye a la calidad del software desarrollado a través de ese atributo. Todos los atributos listados están basados en la norma ISO 25010. A continuación, se presentan los requerimientos no funcionales en las tablas x a y.

% API
\begin{center}
  \begin{tabular}{ | p{2cm}| p{8cm} | p{5cm} |}
    \hline
    \multicolumn{3}{|c|}{\textbf{RNF\_01 (EJEMPLO)}} \\
    \hline
    
    \multicolumn{1}{|p{2cm}|}{\textbf{Descripción}} & \multicolumn{2}{|p{13cm}|}{La plataforma contará con una API REST} \\ \hline
    
    \multicolumn{1}{|p{3.5cm}|}{\textbf{{Atributo}}} & \multicolumn{1}{|p{1.5cm}|}{\textbf{Aplica}} & \multicolumn{1}{|p{10cm}|}{\textbf{Especificación}} \\ \hline
    
    \multicolumn{1}{|p{3.5cm}|}{\nohyphens{Adecuación Funcional}} & \multicolumn{1}{|c|}{X} & \multicolumn{1}{|p{10cm}|}{La API contribuye a la corrección funcional, ya que facilita la obtención de datos precisos del sistema a terceros.} \\ \hline
    
    \multicolumn{1}{|p{3.5cm}|}{\nohyphens{Eficiencia de Desempeño}} & \multicolumn{1}{|c|}{} & \multicolumn{1}{|p{10cm}|}{} \\ \hline
    
    \multicolumn{1}{|p{3.5cm}|}{\nohyphens{Compatibilidad}} & \multicolumn{1}{|c|}{X} & \multicolumn{1}{|p{10cm}|}{La API contribuye a la coexistencia con otras piezas de software independientes, ya que permite a dicho software consumir información del sistema en tiempo real.} \\ \hline
    
    \multicolumn{1}{|p{3.5cm}|}{\nohyphens{Usabilidad}} & \multicolumn{1}{|c|}{} & \multicolumn{1}{|p{10cm}|}{} \\ \hline
    
    \multicolumn{1}{|p{3.5cm}|}{\nohyphens{Fiabilidad}} & \multicolumn{1}{|c|}{X} & \multicolumn{1}{|p{10cm}|}{La API contribuye a la madurez del software, ya que es gracias a ella que el sistema puede satisfacer las necesidades de los usuarios que consumen información del mismo.} \\ \hline
    
    \multicolumn{1}{|p{3.5cm}|}{\nohyphens{Seguridad}} & \multicolumn{1}{|c|}{X} & \multicolumn{1}{|p{10cm}|}{Gracias al diseño de la API, sólo se exponen endpoints de lectura, por lo que esta contribuye a la confidencialidad e integridad de la información.} \\ \hline
    
    \multicolumn{1}{|p{3.5cm}|}{\nohyphens{Mantenibilidad}} & \multicolumn{1}{|c|}{} & \multicolumn{1}{|p{10cm}|}{} \\ \hline
    
    \multicolumn{1}{|p{3.5cm}|}{\nohyphens{Portabilidad}} & \multicolumn{1}{|c|}{} & \multicolumn{1}{|p{10cm}|}{} \\

    \hline
  \end{tabular}

  \captionof{table}{Tabla de requerimiento no funcional de API}\label{table:rnf:api}
\end{center}

\section{Interfaces Internas de Salida}

\begin{center}
	\begin{tabular}{ | c | p{3.5cm} | p{10cm} |}
		\hline
		\textbf{Id} & {\textbf{Nombre}} & {\textbf{Detalle de Datos}} \\ \hline
		{\textbf{IN\_01}} & Modelo X & field\_1, field\_2, field\_3 \\ \hline
		{\textbf{IN\_02}} & Modelo Y &  field\_1, field\_2, field\_3 \\ \hline
	\end{tabular}

    \captionof{table}{Tabla de interfaces internas de salida}\label{table:interfaces:in}
\end{center}

\section{Interfaces Externas de Salida}

\begin{center}
	\begin{tabular}{ | c | p{2cm} | p{6.5cm} | p{4cm} |}
		\hline
		{\textbf{Id}} & 	{\textbf{Nombre}} & {\textbf{Detalle de Datos}} & {\textbf{Medio de Salida}} \\
		\hline
		{\textbf{OUT\_01}} & Modelo X & field\_1, field\_2, field\_3 & Pantalla \\ \hline
		{\textbf{OUT\_02}} & Modelo Y & field\_1, field\_2, field\_3 & Archivo PDF \\ \hline
	\end{tabular}
  
  \captionof{table}{Tabla de interfaces externas internas de salida}\label{table:interfaces:out}
\end{center}


% Análisis Funcional
\chapter{Análisis Funcional}

\section{Actores}
\label{analysis:actors}

La especificación de todos los actores se puede encontrar a continuación en la \autoref{table:analysis:actors}.

\begin{center}
  \begin{tabular}{| p{3cm} | p{4.5cm} | p{4.5cm} | p{2cm} |}
    \hline
    \multicolumn{4}{|c|}{\textbf{Actores}} \\
    \hline
    \multicolumn{1}{|c|}{\textbf{Actor}} & \multicolumn{1}{|c|}{\textbf{Función}} & \multicolumn{1}{|c|}{\textbf{Conocimientos}} & \multicolumn{1}{|c|}{\textbf{Privilegio}}\\
    \hline
    {\textbf{Administrador}} & Cumple con todas las funciones dentro de la empresa, tales como ... & ¿Requiere conocimiento sobre como funciona el sistema en su totalidad? Si no es así, ¿qué tanto? & Máximo. \\ \hline
    {\textbf{Cliente}} & Utiliza la página sólo para visualizar la información que esta ofrece. & No requiere conocimientos técnicos más allá de iniciar sesión. & Ninguno.\\ \hline
  \end{tabular}
  \captionof{table}{Tabla de actores}\label{table:analysis:actors}
\end{center}

\section{Casos de Uso}
Esta sección contiene todos los diagramas de casos de uso relevantes para el proyecto.

\subsection{Diagramas de Casos de Uso}
Imágenes de todos los casos de uso confeccionados para el proyecto.

\subsection{Especificación de los Casos de Uso}

\begin{center}
  \begin{tabular}{| p{1.5cm} | p{6.5cm} | p{5.5cm} |}
    \hline
    \multicolumn{3}{|c|}{\textbf{Casos de Uso}} \\
    \hline
    \multicolumn{1}{|c|}{\textbf{Id}} & \multicolumn{1}{|c|}{\textbf{Actor}} & \multicolumn{1}{|c|}{\textbf{Nombre}}\\
    \hline
    
    % Módulo de Privilegios
    {\textbf{CU\_01}} & \textbf{Administrador, Usuario} & \textbf{Iniciar Sesión}\\ \hline
    
    % Módulo de Autos
    {CU\_02} & {Administrador} & {Crear Modelo X}\\ \hline
  \end{tabular}
  
  \captionof{table}{Tabla de especificación de casos de uso}\label{table:usecase:specification}
\end{center}

\subsection{Detalle de los Casos de Uso}
A continuación, se presentan tablas de detalle para los casos de uso listados en la \autoref{table:usecase:specification} que fueron marcados en negrita. Estos casos de uso también contarán con un detalle de su flujo de eventos básico.

Los casos de uso seleccionados para ser detallados fueron elegidos porque cumplen funciones fundamentales de los requisitos funcionales de la aplicación desarrollada. El resto de los casos de uso solamente contarán con sus precondiciones y una descripción simple.

%
% Privilegios
%

\begin{center}
  \begin{tabular}{| p{7.5cm} | p{7.5cm} |}
    \hline
    \multicolumn{2}{|p{15cm}|}{\textbf{CU\_01\_INICIAR\_SESION} (Usuario)} \\ \hline
    \multicolumn{2}{|p{15cm}|}{\textbf{Pre-Condiciones:} El usuario debe estar en la página web. El usuario debe haberse registrado en la plataforma.} \\ \hline
    \multicolumn{2}{|p{15cm}|}{\textbf{Post-Condiciones:} El usuario inicia sesión en la plataforma.} \\ \hline
    \multicolumn{2}{|p{7.5cm}|}{\textbf{Flujo de Eventos Básicos}} \\ \hline
    \multicolumn{1}{|p{7.5cm}|}{\textbf{Usuarios:} Administrador, Usuario} & \multicolumn{1}{|p{7.5cm}|}{\textbf{Sistema}} \\ \hline
    
    \multicolumn{1}{|p{7.5cm}|}{} & 
    \multicolumn{1}{|p{7.5cm}|}{1. Renderiza la pantalla de inicio de sesión.}\\ \hline
    
    \multicolumn{1}{|p{7.5cm}|}{2. Ingresa su correo y contraseña, y luego pulsa el botón para iniciar sesión.}& 
    \multicolumn{1}{|p{7.5cm}|}{3. Valida la información ingresada por el usuario.}\\ \hline
    
    \multicolumn{1}{|p{7.5cm}|}{} & 
    \multicolumn{1}{|p{7.5cm}|}{4. Sesión iniciada. Redirecciona al usuario a la página principal.}\\ \hline
    
    \multicolumn{2}{|p{7.5cm}|}{\textbf{Flujo de Eventos Alternativo}} \\ \hline
    
    \multicolumn{1}{|p{7.5cm}|}{\textbf{Usuarios:} Administrador, Organizador, Moderador y Jugador} & \multicolumn{1}{|p{7.5cm}|}{\textbf{Sistema}} \\ \hline
    
    \multicolumn{1}{|p{7.5cm}|}{} & 
    \multicolumn{1}{|p{7.5cm}|}{3 (b). Si las credenciales son incorrectas, el sistema muestra un mensaje de error.}\\ \hline
    
    \multicolumn{1}{|p{7.5cm}|}{} & 
    \multicolumn{1}{|p{7.5cm}|}{4 (b). Vuelve al paso 2 del flujo básico.}\\ \hline
  \end{tabular}
  
  \captionof{table}{Tabla del caso de uso CU\_01\_INICIAR\_SESION}\label{table:usecase:1}
\end{center}

\begin{center}
  \begin{tabular}{| p{7.5cm} | p{7.5cm} |}
    \hline
    \multicolumn{2}{|p{15cm}|}{\textbf{CU\_02\_CREAR\_MODELO\_X} (Administrador)} \\ \hline
    \multicolumn{2}{|p{15cm}|}{\textbf{Pre-Condiciones:} El administrador debe haber iniciado sesión con sus credenciales.} \\ \hline
    \multicolumn{2}{|p{15cm}|}{\textbf{Descripción:} El sistema guarda el modelo X en la base de datos con los datos ingresados por el administrador. } \\
    \hline
  \end{tabular}
  
  \captionof{table}{Tabla del caso de uso CU\_02\_CREAR\_MODELO\_X}\label{table:usecase:3}
\end{center}

\section{Modelo de Datos}
Modelos de la base de datos. Pueden ser esquemas de SQL, diagramas para bases de datos no relacionales, etc.

\section{Esquema de la Base de Datos}
Esquemas de definición para los modelos de la base de datos. Por ejemplo, pueden ser los modelos escritos en JSON, directamente en  en algún lenguaje de programación, etc.

\section{Diseño de Interfaz}
En esta sección se presentan imágenes de mockups o capturas de pantalla del software terminado, para así ilustrar las interfaces realizadas.

\subsection{Paleta de Colores y Tipografía}
Imágenes o descripción de la paleta de colores y tipografía utilizadas.

\section{Diseño de Arquitectura}
Describir el diseño de la arquitectura utilizada para montar el proyecto desarrollado. Por ejemplo: ''El proyecto, en su estado actual, hace uso de un servidor propio, el cual contiene los servicios web, bases de datos y caché, todo en una sola máquina ...''

\section{Estructura del Código}
El proyecto es una aplicación hecha en el framework X, por lo tanto sigue el patrón Y...

Insertar alguna imagen o tabla que permita visualizar el árbol de directorios/archivos del proyecto.

\subsection{Estándres de Codificación}

\begin{itemize}
  \item Toda la base de código y documentación debe estar escrita en inglés.
  \item Se utilizan linebreaks (EOL - End of Line) CRLF.
  \item Se utilizan dos espacios para la indentación del código, no tabulaciones.
  \item La codificación del proyecto es en UTF-8. 
\end{itemize}

\subsection{Backend}
La \autoref{table:backend} entrega una especificación de los directorios relevantes para el backend del proyecto.

\begin{center}
  \begin{tabular}{ | l | p{12.5cm} |}
    \hline
    \multicolumn{1}{|c|}{\textbf{Directorio}} & \multicolumn{1}{|c|}{\textbf{Detalle}} \\
    \hline
    
    {\textbf{controllers}} & Contiene todos los controladores ... \\ \hline
    
    {\textbf{models}} & Contiene todas las clases que modelan y envuelven los datos almacenados en la base de datos de la aplicación ... \\ \hline
  \end{tabular}
  
  \captionof{table}{Tabla de directorios del backend del proyecto}\label{table:backend}
\end{center}

\subsection{Frontend}
La \autoref{table:frontend} entrega una especificación de los directorios relevantes para el frontend del proyecto.

\begin{center}
  \begin{tabular}{ | l | p{12.5cm} |}
    \hline
    \multicolumn{1}{|c|}{\textbf{Directorio}} & \multicolumn{1}{|c|}{\textbf{Detalle}} \\
    \hline
    
    {\textbf{views}} & Contiene todas las vistas de la aplicación ... \\ \hline
    
    {\textbf{assets}} & Todas las imágenes, hojas de estilo ... \\ \hline
  \end{tabular}
  
  \captionof{table}{Tabla de directorios del frontend del proyecto}\label{table:frontend}
\end{center}


% Training
\chapter{Plan de Capacitación, Implantación y Puesta en Marcha}

\section{Estado del Proyecto}
Describir el estado del proyecto y su relación con el plan de capacitación. Por ejemplo:

''Actualmente, el proyecto se encuentra finalizado. Esto quiere decir que todos los módulos, requerimientos funcionales y objetivos propuestos han sido alcanzados con éxito.

Si bien el software del proyecto está completamente finalizado, éste aún no pasa a ser utilizado por el público, por lo que sigue estando en una etapa de puesta en marcha ...''

\section{Implantación y Puesta en Marcha}
...

\section{Plan de Capacitación}
Para este proyecto, se ha considerado un plan de capacitación que consiste en ...


% Conclusión
\chapter{Conclusión del Proyecto}
Aquí se concluye sobre los objetivos del proyecto, si es que estos se lograron, en qué aportaron al perfil del tesista, etc ...

\begin{itemize}
	\item ...
	\item ...
	\item ...
	\item ...
	\item ...
\end{itemize}


% Anexos
\chapter{Anexos}
En este capítulo se listan elementos relacionados directamente con la confección del presente informe y con el desarrollo del software que se ha realizado.

\section{Anexo Estimación de Casos de Uso}
En esta sección se evalúan los factores de complejidad técnica y ambiental. La \autoref{table:tcf} es el detalle utilizado para obtener el ''Technical Complexity Factor'', o ''TCF'', y la \autoref{table:ef} el detalle de ''Environment Factors'', o ''EF''.

\begin{center}
  \begin{tabular}{ | p{4cm} | p{2cm} | p{4cm}| p{4cm} | } 
    \hline
    \multicolumn{4}{|c|}{\textbf{Technical Complexity Factor (TCF)}} \\
    \hline
    \multicolumn{1}{|p{3cm}|}{\textbf{Technical Factor}} & \multicolumn{1}{|c|}{\textbf{Multiplier}} & \multicolumn{1}{|c|}{\textbf{Relevancia Percibida}} & \multicolumn{1}{|c|}{\textbf{Resultado Multip.}} \\
    \hline
    
    {\textbf{Distributed System}} & 2 & 2 & 4 \\ \hline
    {\textbf{Application performance objectives, in either response or throughput}} & 1 & 2 & 2 \\ \hline
    {\textbf{End-user efficiency (on-line)}} & 1 & 3 & 3 \\ \hline
    {\textbf{Complex internal processing}} & 1 & 3 & 3 \\ \hline
    {\textbf{Reusability, the code must be able to reuse in other applications}} & 1 & 3 & 3 \\ \hline
    {\textbf{Installation ease}} & 0,5 & 1 & 0,5 \\ \hline
    {\textbf{Operational ease, usability}} & 0,5 & 2 & 1 \\ \hline
    {\textbf{Portability}} & 2 & 3 & 6 \\ \hline
    {\textbf{Changeability}} & 1 & 3 & 3 \\ \hline
    {\textbf{Concurrency}} & 1 & 2 & 2 \\ \hline
    {\textbf{Special security features}} & 1 & 3 & 3 \\ \hline
    {\textbf{Provide direct access for third parties}} & 1 & 0 & 0 \\ \hline
    {\textbf{Special user training facilities}} & 1,5 & 4 & 6 \\ \hline
  \end{tabular}
  
  \captionof{table}{Tabla de complejidad técnica}\label{table:tcf}
\end{center}

Se obtiene entonces que el total es 34, con lo que sustituyendo en la fórmula del TCF, quedaría:
\[
\text{TCF} = 0.6+(0.01\cdot34)
\]

\[
TCF = 0.94
\]

\begin{center}
  \begin{tabular}{ | p{4cm} | p{2cm} | p{4cm}| p{4cm} | } 
    \hline
    \multicolumn{4}{|c|}{\textbf{Environment Factors (EF)}} \\
    \hline
    \multicolumn{1}{|p{3cm}|}{\textbf{Environmental Factor}} & \multicolumn{1}{|c|}{\textbf{Multiplier}} & \multicolumn{1}{|c|}{\textbf{Relevancia Percibida}} & \multicolumn{1}{|c|}{\textbf{Resultado Multip.}} \\
    \hline
    
    {\textbf{Familiar with Iterative Methods}} & 0,5 & 5 & 2,5 \\ \hline
    {\textbf{Application experience}} & 1 & 5 & 5 \\ \hline
    {\textbf{Object Oriented experience}} & 0,5 & 5 & 2,5 \\ \hline
    {\textbf{Analyst capability}} & 1 & 5 & 5 \\ \hline
    {\textbf{Motivation}} & 2 & 5 & 10 \\ \hline
    {\textbf{Stable requirements}} & -1 & 0 & 0 \\ \hline
    {\textbf{Difficult programming language}} & -1 & 3 & -3 \\ \hline
  \end{tabular}
  
    \captionof{table}{Tabla de factores medioambientales}\label{table:ef}
\end{center}

\section{Anexos de Recopilación de Información}
...

\section{Anexo Aspectos de Gestión de Proyectos}
...

\subsection{Anexo Resumen de Esfuerzo}
\begin{center}
  \begin{tabular}{ | p{10cm} | p{5cm} |}
    \hline
    \multicolumn{1}{|c|}{\textbf{Actividad}} & \multicolumn{1}{|c|}{\textbf{Número de Horas}} \\
    \hline
    
    {Preparación del proyecto} & {20} \\ \hline
    {Desarrollo del módulo de autos} & {50} \\ \hline
    {Desarrollo del módulo de pistas} & {50} \\ \hline
    {Desarrollo del módulo de sesiones} & {80} \\ \hline
    {Desarrollo del módulo de temporadas} & {80} \\ \hline
    {Corrección de errores de código} & {40} \\ \hline
    {Despliegue de la aplicación} & {27} \\ \hline
    {Control de versiones} & {37} \\ \hline
    
    {\textbf{Total}} & {\textbf{384}} \\

    \hline
  \end{tabular}
  
    \captionof{table}{Tabla de resumen de esfuerzo}\label{table:effort}
\end{center}

\section{Anexos Retrospectiva del Proyecto}

\subsection{Anexo Iteraciones en el Desarrollo}

\begin{center}
  \begin{tabular}{ | p{6cm} | p{3cm} | p {6cm} |}
    \hline
    \multicolumn{1}{|c|}{\textbf{Funcionalidad}} & \multicolumn{1}{|c|}{\textbf{Fecha}} &
    \multicolumn{1}{|c|}{\textbf{Retroalimentación}} \\
    \hline
    
    {Módulo X (1)} & {12/11/2023} & {Añadir visualización para todos y uno sólo ...}\\
    {Módulo X (2)} & {12/11/2023} & {Agregar a la navegación ...}\\
    {Módulo X (3)} & {21/11/2023} & {Corregir problemas internos del módulo X}\\ \hline
  \end{tabular}
  
   \captionof{table}{Tabla de iteraciones en el desarrollo}\label{table:iterations}
\end{center}


% Bibliografía
\bibliographystyle{ieeetr}

\bibliography{main}

\end{document}
